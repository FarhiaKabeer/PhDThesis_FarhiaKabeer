
{\chapter{Conclusions and discussion}

}
\label{ch:Chapter6}

Predicting the dynamics of malignant cells in cancers and deconstructing the molecular basis of resistance and metastasis can be achieved by repeated observation of cell populations with shared genotypes and phenotypes. In this thesis project, I have applied this concept to show that cell population dynamics in cancer can be resolved by assigning cells to specific clones, defined by copy number. I repeatedly observed similar fitness coefficients for the same copy number genotypes, showing deterministic behaviour. The fitness differences between most clones is small, despite large changes in clone prevalence over time and they have the propensity to regulate gene expression in some instances. 


\section{Identification of drug response properties and methods of tissue preparation for single cell measurements in patient derived xenografts}

In Chapter 3, I have demonstrated the effectiveness of the 3x1x1 experimental paradigm for examining population-based \textit{in vivo} quick compounds screens. It enabled insight into inter-patient response heterogeneity in an efficient manner, and helped to identify quick responsive subpopulations, thus enabling the selection of potential breast tumors or drugs. However, while testing new compounds, it would be more reliable to increase the number of mouse per experiment per drug. It highly depends on the scientific question that need to be answered. 
The technical problems using only 3 PDX could be that sometimes one tumor behaves like an outlier, growing very fast or very slow. 
%Moreover, we could lose one or two tumours. To address this issue, I had to originally transplant more than 3 mice per experiment per drug to my experiments work. 
\\
As with other preclinical models, there were limitations with PDX,
including the lack of an intact immune system, differential control
of mouse stroma versus human stroma and the under representation of specific genotypes and lineage subtypes. However, major genetic alterations found in patient tumors corresponded in the PDX. Also, \textit{Eirew et al.} showed that clonal dynamics during \textit{in vivo} engraftment in breast cancer patient xenografts, are genetically relatively stable \cite{eirew2015dynamics}.
\\
In addition, the patterns of drug sensitivities that I found \textit{in vivo} screens were highly correlated with the breast cancers. All TNBC were sensitive to cisplatin consistent with the choice of DNA damaging agents as a first line of cancer therapy in triple negative breast cancers. There is renewed interest in treating TNBC with cisplatin. At present, $\sim$ 20 clinical trials are exploring the benefits of cisplatin to treat TNBC either as a single agent or in combination with other therapies \cite{hill2019cisplatin}. Moreover, I found that the TNBC PDX tumors that possess fold back inversion (FBI) background were resistant to PARP inhibitors. This finding was also consistent with the fact that Ovarian tumors with FBI signatures would be less responsive to PARP inhibition \cite{wang2017genomic}.
   
   
Current sequencing techniques require single-cell suspensions for passage through microfluidic platforms, and the generation of single-cell suspensions from solid tissues requires the enzymatic and mechanical disruption of extracellular matrix and cell-cell contacts. Moreover, during both dissociation of tissues and passage through fluidic devices, cells can undergo stress, shearing, anoikis, and apoptosis \cite{aljanahi2018introduction}. For this reason, efforts must be made on both sample handling and bioinformatics to ensure minimal noise and optimal filtration of data.
\\
Predominantly, I identified a subset of dead cells which would not be filtered by standard practices. I reported that cells that were FACS sorted as either live, dying, or dead (with/without induction of cell death with TNF-alpha) were present in all three clusters, emphasizing that although the transcriptomic state and surface markers of the cell are correlated, they are not the same. Such concepts are implicative  of ``pseudotime'' in single-cell trajectory analysis, whereby developmentally ordering cells transcriptomically can lead to early or late cells being placed at variable positions along the pseudotime trajectory \cite{campbell2018descriptive, campbell2018uncovering}. 
 \\
 I identified a conserved collagenase-associated transcriptional pattern including induction of stress and heat shock genes, consistent with a transcriptional response identified in a subset of muscle stem cells \cite{van2017single}, and which was minimized when samples were dissociated at cold temperatures with a cold active serine protease. This observation has clinical implications because chemotherapies are also known to induce stress response. For example, \textit{JUN} and \textit{FOS} are associated with cancer drug resistance and metastatic progression \cite{insua2018stress,fan2017ap, ramsdale2015transcription}. Moreover, heat shock proteins (HSPs) are under study as the potential therapeutic targets for anti-cancer treatment \cite{yun2020heat}.
 It would be interesting to tease apart the technical effects of tissue handling on heat shock proteins (HSPs) versus the real effects induced by chemotherapies. This could be done with more control samples and testing various conditions.
I suggested that each tissue and dissociation method should be assessed for dissociation-induced signatures before undertaking large-scale scRNA-seq experiments. 
 

\section{Copy number could be associated with clonal fitness in human breast cancers}
Decoding the contributions of clonal competition in the course of tumour growth with and without drug selection can be achieved by measuring and modeling cellular dynamics with granular scWGS timeseries.

In Chapter 4, I have shown for the first time how \textbf{copy number fitness} is distributed in human cancers across clones and identified important mechanisms, such as fitness cost. 

I observed clonal expansions in human breast cancer cells through serial passaging, single cell sequencing, and fitness modeling in unperturbed patient derived xenografts. 
All series exhibited progressively higher tumour growth rates over time. In contrast to HER2+SA532, all TNBC PDX models exhibited evidence of clonal diversification associated with copy number alterations and selection coefficients consistent with positive selection over time \textbf{(\autoref{fig:landscapefitness})}. This implies selection mechanisms operate in the growing tumors even without drug pressure, ultimately leading to CNA mediated clonal fitness as described. 

Next, I tested how cisplatin impacted the stability of the fitness landscape of the three independent TNBC PDX series. 
I found that continuous drug exposure resulted in suppression of clones that dominated in the absence of treatment and the emergence of drug resistance on the background of low fitness and/or previously unobserved CNA genotypes. This concept of cancer evolution under drug pressure is a key determinant of patient outcomes across all human cancers. 
 
Only two drugs were studied, Cisplatin and CX5461, and only one in detail. Whether similar fitness inversions would be observed with other drugs is unknown.  
%This is because I have shown with reproducible clonal dynamics that the drug resistant clone arose from the same selected clade.
However, there is evidence in literature that clonal selection ultimately results in the relative outgrowth of the selected drug resistant clone within the tumours. Observations of copy number progression over time suggests that clonal selection via copy number progression could be a feature of other therapies. \cite{aparicio2013implications, graham2017measuring, wu2012clonal, liu2009copy}. 

I showed that resistance to platinum, in polyclonal cancers, was associated with a \textbf{fitness cost}, interpreted from an early reversibility in treatment. This concept has clinical implications and could be exploited in future therapeutic strategies. In PDX, during early treatment, when drug was withdrawn, the tumor clonal composition reversed back to sensitive clones. Some clinical trials for chemotherapy rechallenge in patients are also underway. \cite{neuzillet2016platinum, cremolini2019rechallenge}.
Hence, forecasting the trajectories of cancer clones is of immediate importance to understanding therapeutic response in cancer and for deploying adaptive approaches \cite{Vasan2019-mt}.

The presence within a tumour of lineage precursors to resistant genotypes may define time windows within which clonal competition could mediate plasticity to treatment. The only possibility to catch this time window is to track the developing tumor with serial sampling. In PDX timeseries, I showed that the clonal composition evolved overtime. But in patients, it is not very convenient, or rather impossible if the tumor is growing at an ectopic site, to do timeseries tumor sampling. However, serial measurements of circulating tumour DNA (ctDNA) can allow the evolutionary dynamics of cancers to be tracked non-invasively . It is already established that detection of ctDNA in plasma of breast cancer patients is possible \cite{beaver2014detection, garcia2015mutation, merker2018circulating} and if this is tracked serially, clonal evolutionary dynamics could be detected early during treatment. 
At present, the efforts in studying ctDNA are more towards detecting tumor burden and treatment response, however, \textbf{Wright-Fisher} model, as used in  Chapter 4, can be useful to apply on serial ctDNA samples.

 Further studies with timeseries modeling will provide insight into therapeutic strategies promoting early intervention and evolution-aware approaches to clinical management \cite{Acar2020-tf}.
 

\section{Clone specific genotypes drives clone-specific gene expression programs}

After establishing the fitness of genomic clones in the presence or absence of chemotherapies in Chapter 4, I next sought to uncover the basis of how copy number variation alters gene expression in complex cell mixtures. In parallel, I sequenced single-cell RNA from independent cell populations and mapped for genome-transcriptome association. 

I revealed that gene expression states can be assigned to cancer clones using single-cell RNA and DNA sequencing independently sampled from heterogeneous cell populations. 
I also demonstrated that in order to study cancer evolution, tumor samples have to be collected at various time points, which could, however, create bias in RNA sequencing libraries. I successfully minimized this bias by applying \texttt{SCTransform} normalization and batch correction techniques. Nevertheless, residual batch effects were observed in dimension reduction. This implies that single cell RNA-seq data requires special pre-processing to counter batch effects for better biological interpretations. Also, the residual batch effects emphasized the need to validate results. The common ways of validation are immunohistochemistry (IHC) and western blots for the detected proteins. 

%--------------------------------------------------------------------
In addition, I was able to dissect the regulation status of differentially expressed genes between genomically defined resistant and sensitive clones in all TNBC PDX series. 
I demonstrated that $\sim$ 7\% of the cisplatin resistance related gene expression was associated with positive linear tendency with CNA. That means upregulation or down regulation of gene expression was mediated by copy number gain and loss, respectively. However, some 
showed paradoxical behaviour with copy number (negative tendency). 

Moreover, differentially expressed genes and pathways between resistant and sensitive clones confirmed known cisplatin resistance mechanisms as explained in the introduction. For example, \textbf{OXPHOS} is known to be upregulated in cisplatin related drug resistance mechanisms \cite{lee2017myc}, as well as \textbf{Apoptosis} \cite{panaretakis2012cisplatin}, \textbf{TGF Beta signaling} \cite{zhang2019tgfbeta1}, \textbf{Estrogen response} \cite{zhu2018er}, \textbf{E2F targets} \cite{zheng2020upregulation}, \textbf{TNFA signaling via NF-$\kappa$B} \cite{lagunas2008nuclear,ito2015down, ryan2019targeting} and \textbf{Hypoxia} \cite{lee2012hypoxia, mcevoy2015identifying, deben2018hypoxia,li2019erk}.
 \textbf{Interferon alpha response} \cite{provance2019deciphering} and \textbf{Interferon gamma response} \cite{mojic2018dark} were mainly upregulated with increasing doses of cisplatin in SA1035. Since SA1035 has APOBEC mutational signatures, perhaps these pathways upregulations are the consequence of enhanced intrinsic immune response that could benefit from immunotherapy. Interferon pathways are not well studied in reference to cisplatin and could behave as potential targets for circumventing cisplatin resistance.
 
 %Other pre, post and on-target known mechanisms of cisplatin resistance genes, matched with differentially expressed genes of resistant versus sensitive clones, including \textbf{copper transporting, beta polypeptide (ATP7B)}, \cite{katano2002acquisition,komatsu2000copper, aida2005expression}, high expression of the \textbf{high-mobility group box protein 1 (HMGB1)}, that appears to bind selectively to cisplatin DNA crosslinks and interferes with nucleotide excision repair (NER) \cite{awuah2017repair, mukherjee2019targeting}, upregulation of \textbf{IGF1R} expression  
 %\cite{selfe2018igf1r}, \textbf{voltage dependent anion channel (VDAC)} \cite{yang2006cisplatin}. 
 
 However, the main purpose of my research here is to identify how many of the resistance causing genes are operated through the underlying changes in genomes and how many of them are independent. This concept is clinically important because \textit{in trans} regulated genes can be probed to circumvent cisplatin resistance, but if the change in gene expression is associated with fixed genomic changes, it can not be reversed. For example,  high expression of the \textbf{high-mobility group box protein 1 (HMGB3), (known for on-target cisplatin resistance)}, appears to bind selectively to cisplatin DNA crosslinks and interferes with nucleotide excision repair (NER).
 Literature says that cisplatin chemosensitivity can be achieved by targeting \textbf{HMGB3} gene expression \cite{awuah2017repair, mukherjee2019targeting}. However, it was regulated \textit{in cis}, in SA609 and \textit{in trans}, in SA535, which implies that cisplatin chemosensitivity can only be regained in SA535 but not in SA609, by targeting \textbf{HMGB3}. Similarly, \textbf{voltage dependent anion channel (VDAC)} (known for on-target cisplatin resistance) \cite{yang2006cisplatin} was regulated \textit{in cis}, in SA609 and \textit{in trans}, in SA535.
 
 Furthermore, \textit{in cis} regulated genes can be explored as predictive biomarkers of prognosis and disease progression. For example, {\textbf{CRABP2, Peroxisome pathway}} was an upregulated gene \textit{in cis} in resistant tumor, and it has not been studied as a cisplatin resistance gene in breast cancer. However, it is well established that overexpression of this gene promotes \textbf{epithelial mesenchymal transition (EMT)}, metastasis and invasion of triple negative breast cancer cells by inactivating the Hippo pathway \cite{feng2019crabp2}.

\section{Limitations and future directions}

In Chapter 3, the experimental design of 3x1x1 could be improved to increase the number of mice per experiment. Increasing the number might change the interpretation of results from partially sensitive tumors to full sensitive. 

ScRNA-seq analysis data provided a comprehensive set of information regarding tissue handling and its effects on sequencing analysis and biological interpretation. 
However, stress is also induced by chemotherapies. More replicates as well as untreated controls would help differentiate between the stress response generated by technical handling.

A limitation of using cold active protease enzyme is that  simultaneous \ac{DLP+} libraries alongside scRNA-seq were not compared at different temperatures. In the future, I plan to directly investigate the difference of single cell whole genome sequencing, by comparing digestion at 6\textdegree C and at 37\textdegree C. In addition, while performing  experiments for Chapter 3, I noticed that cell yield at at cold temperature was lower as compared to digestion at 37\textdegree C. This factor plays an important role while make multiple measurements from the same tumor. Moreover, the type of tissue, and its consistency and exposure to either chemotherapy or environmental and genetic perturbations, should also be considered as factors. The choice of digestion enzyme and temperature should be optimized accordingly before conducting any large scale study.

In Chapter 4, I applied phylogenetic model on breast cancer timeseries data but in reality, cancer cells have the propensity to violate the rules of perfect phylogeny. This is mainly because CNA-change points, i.e. markers \textbf{breakpoints (points of copy number change)} are used instead of integer CNA states as phylogenetic traits. Non-overlapping CNA events do not violate the phylogeny assumptions, but the overlapping events could cause violation as explained in Chapter 1. 

%Also, branch lengths could not be measured through the model that implies a need to modify the model in future with clock assumptions.

I recognized that establishing causal relationships based on copy number clustering into clones is difficult and requires more controlled experiments and larger sample sizes. However, exploring the potential correlates of differential fitness may aid in designing follow up experiments. The untreated dynamics were confirmed by two different sets of mixture experiments. In the future, more mixture experiments for the timeseries that undergo neutral dynamics will explain the tumor biology explicitly over time. 

Moreover, in the context of pharmacological intervention, I acknowledged that the drug exposure started from early time points that derives the competition among the clones and infers the fitness of certain clones. It would be more interesting and comparable to conduct experiments where later passages of the time series PDX are taken and chemotherapy introduced in the same way. Clonal competitions and fitness in that environment will become obvious.

The future plan is to create more sequencing depth for SNV detection by pseudobulk, in detail. All the data is being processed and under investigations for SNVs, detailed mutational analysis and integration of allelic imbalance information caused by clone-specific \ac{LOH} events. Follow up replication experiments sequenced at higher depth may be required to rule out the existence of driving mutations. 

Finally, to make the findings generalizable, more TNBC tumors would be required in the same experimental settings.


In Chapter 5, scRNA-seq helped identify the transcriptomes of individual cells but it did not exactly explain relationships between the cells. For that,  co-detection of transcripts and proteins for mRNA, and signaling network relationships, would be required in solid state tumors by image mass cytometry (IMC).


%the focus of \texttt{clonealign} has been on linking transcriptional measurements to genomically defined clones assuming only a copy-number dosage effect on transcript abundance. While the \texttt{clonealign} model also allows for integration of allelic imbalance information caused by clone-specific \ac{LOH} events. That could be further probed in future. 
%However, full-transcript-length single-cell RNA sequencing technologies such as Smart-seq2 \cite{picelli2014full} would allow for further refinement of clonal assignment and represent the appropriate use of clonealign's incorporation of allelic imbalance information.

Moreover, I explored single cell expression in reference to resistant and sensitive clones and identified some new pathways and genes activated in TNBC that are not already well studied in context of cisplatin resistance in breast cancers. In the future, I plan to validate and explore them as new resistance mechanisms.


%I have the data of genes that are regulated \textit{in cis} or \textit{in trans}. If the gene is upregulated \textit{in trans}, it could be studied as a potential tumor biomarker and gene editing through CRISPRi \cite{larson2013crispr}, for example could be potential follow up experiments.
%Other future experiments would be to use CRISPR/cas9 \cite{doudna2014new} technology for biomarker validations. However, future experiments would also involve the validations through imaging mass cytometry (IMC). 

Another supporting biological methodology would involve epigenetic changes, which would influence the fitness of clones. It can be done by interrogating the system by chromatin accessibility and/or DNA methylation assays . It can further validate scRNA expression in context of measuring gene promoters hypermethylation or hypomethylation.
%of tumor suppressor genes  or hypomethylation of oncogenes.









 
 %Single cell RNA seq analysis data can help in a better comprehension of some widespread and more specific biological mechanisms involved in breast cancer progression.that could be further validated pertaining to drug resistance in triple negative through modern techniques, such as, CRISPR/Cas9.

%Genetic alterations and the contextual signaling by the microenvironment give rise to an intricate network of pathological mechanisms and molecular pathways


%Individual tumors and cancer cells exhibit substantial molecular diversity.

%Single cell RNA seq analysis data can help in a better comprehension of some widespread and more specific biological mechanisms involved in breast cancer progression.