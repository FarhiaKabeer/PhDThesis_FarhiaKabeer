
%The important features of a timeseries patient-derived xenografts  that differentiate it from cross-sectional data-collection procedures \cite{kaplan1995time}:
 
 %\textbf{(a)} Repeated measurements of a given behavior are taken across time at equally spaced intervals.Taking multiple measurements is essential for understanding any given behavior is forced to evolve over time, and doing so at equal intervals give an open opportunity to clearly investigate the dynamics of that behavior manifesting at distinct time scales.
 %\textbf{(b)} The temporal ordering of measurements is preserved. Doing so is the only way to fully examine the dynamics governing a particular process. If we expect that a specific event will influence the dynamics of clones in a particular way, utilizing summary statistics will completely ignore the temporal ordering of the data and likely occlude one’s view of important behavioral dynamics.





Dynamics of single cell genomes and transcriptomes in response to chemotherapies

Chapter 1. To discern the stability and reproducibility of drug response properties  
1.1. To understand the range of drug response of breast cancer cell lines with respect to DNA repair deficiency- exploring NHEJ pathway in vitro for CX-5461 mechanism-(co-author paper published in Nature communication)
1.2. Large scale screening of CX-5461 in pooled cancer cell lines to discover genotype-specific vulnerabilities (on going at Broad institute)
1.3. To understand the range of drug response of breast cancer PDX in vitro
1.4. Optimization of PDX tumor processing and freezing techniques for SC whole genome, SC PBAL and SC RNA sequencing. (co-author 3 manuscripts on BioRxiv) 
1.5. Identification of tissue handling on sc-RNA seq- (co-first author manuscript in prep.)
1.6. Dissecting the maximum tolerable drug doses in NRG mice?
1.7. What are the basic patterns of drug sensitivity in PDX in vivo? (on-going)

Chapter 2. Characterizing the fitness of cancer cell in time and space

Understanding the evolutionary process within a tumor (Co-first author- “Fitness” manuscript in preparation)
1.1. Estimating the quantitative fitness of clones from time series sampling of PDX without perturbation
1.2. Do we see the same clones evolving if we physically mix early and late passages at two different dilutions?
1.3. Can we observe any change in clonal dynamics with drug perturbation? (Taxol on SA609)
1.4. What are the growth dynamics of the serially passaged PDX tumors?
1.5. Describing phenotype of all passages of PDX tumors through histological staining and expression at certain time points (SC-RNA-seq) 

Chapter 3. To understand the convergence or divergence of clonal dynamics under chemotherapeutic drugs’ selection (“lineage under drug selection” manuscript outlined)
1.1. Estimating the quantitative fitness of clones from time series sampling of PDX with drug perturbation (SA604, SA609, SA535, SA1035)
1.2.  Do the same clones mediate drug resistance or sensitivity over multiple instances of the same PDX (Using genomic markers of lineage at single cell level)
1.3. Can we compare the lineages to be compared when defined by CNA, SNV or rearrangements (Phlylogenetic tree presents same branching?) 	
1.4. To dissect the relationship between co-existing genomes and transcriptomes with chemotherapies
1.5. Can we discover new biomarkers for new targeted compounds (CX-5461) from time series sampling of PDX (CX-5461 in SA535) 

