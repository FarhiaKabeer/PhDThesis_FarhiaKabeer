\section*{Results}
\subsection*{Modeling clonal fitness and selection}
We developed an experimental and computational platform consisting of three major components: scaleable phylogenetics for single cell genomes to identify clones, timeseries sampling of immortal cell lines ($>$10 months) and patient derived xenografts ($>$2.5 years) to observe clonal dynamics, and a mathematical model for inferring clone-specific fitness measures  (\reffig{fig:schematic}).  For normal human breast epithelial cells\cite{Burleigh2015-fk} \emph{in vitro} and in breast cancer PDX\cite{Eirew2015-sg,Bruna2016-ie} (\reffig{fig:schematic}A), we sequenced $>$60,000 cells over interval passaging (\reffig{fig:schematic}B, \refsuptab{tab:omnibusMedians}) with single cell whole genome sequencing\cite{Laks2019-dm} (scDNA-seq) measuring single cell copy number profiles and computing phylogenetic trees to identify genotypic clones and their relative abundances as a function of time. Genetic (p53 biallelic inactivation for cell lines) and pharmacologic (cisplatin dosing in PDX models) perturbations were applied to determine their impact on fitness landscapes. In conjunction with scDNA-seq, we generated sc-RNAseq from $>$47,000 cells to establish clone-specific phenotypes (\reffig{fig:schematic}C, \refsuptab{stab:tenx}), estimated through clone-specific expression profiles. Timeseries clonal abundance observations were modeled using an implementation of the Wright-Fisher diffusion process (\reffig{fig:schematic}D) we called \textit{fitClone} (Supplementary Information). FitClone simultaneously estimates growth trajectories and fitness coefficients for each clone in the population (\reffig{fig:schematic}E). The model accounts for drift as well as selection, with fitness estimated relative to a reference population. As a generative process, the model can be used for forecasting evolutionary trajectories of specific clones. 

We carried out simulation experiments to establish the theoretical behaviour and limitations of \textit{fitClone}, over a range of parameters. Model fits were robust to effective population size and number of clones (\refsupfig{sfig:simul}B,C). In two key advances, simultaneous modeling of multiple clones was superior to modeling each clone independently (\refsupfig{sfig:simul}A), and accounting for stochasticity in the dynamics via the diffusion process led to more accurate selection coefficient estimates than with a deterministic growth model (\refsupfig{sfig:simul}A). Together these simulations established a rationale for systematic modeling of all clones in a unified approach with a generative process. We then applied \textit{fitClone} to previously published data, wherein experimentally derived reproducible clonal dynamics were reported in breast cancer PDXs\cite{Eirew2015-sg}. From one of these lines, the abundance of five major clones (A-E), as determined by single cell genotyping was measured over six serial passages. One clone (clone E) was found in the original study to undergo a selective sweep in repeat and independent passaging\cite{Eirew2015-sg}, suggesting selection due to higher fitness. \textit{fitClone} estimates converged with Clone E bearing the highest fitness (1+s=1.03 $\pm$ 0.01), consistent with positive selection over the timeseries (\reffig{fig:schematic}E), and thus representing a proof of principle application of \textit{fitClone} on real-world data. 

\subsection*{Segmental aneuploidies drive positive selection in diploid p53 deficient cells}
We next applied our framework to immortalized 184hTert diploid breast epithelial cell lines\cite{Burleigh2015-fk} to determine mechanisms by which \textit{TP53} mutation induces clonal expansions and fitness trajectories. \textit{TP53} is the most abundantly mutated gene in all human cancers\cite{Vega2018-hs}, and specifically in breast cancers\cite{Cancer_Genome_Atlas_Network2012-ug,Shah2012-vx}.  Known to be permissive of genomic instability, \textit{TP53} loss is often acquired early in evolution and results in profound alteration of the copy number landscape\cite{Patch2015-xq,Li2020-cn,Nik-Zainal2016-tu,McPherson2016-nv}. We therefore asked whether specific clonal expansions could be observed, and moreover if selective fitness advantages could be quantified, as a function of \textit{TP53} ablation, thereby modeling copy number driven etiologic processes in a controlled system of immortalized mammary epithelial cells. \textit{TP53} wildtype (\textit{p53 WT}) timeseries sampling (60 passages over 300 days, 4 samples) was contrasted with two isogenic \textit{TP53} deficient \verb|NM_000546(TP53):c.[156delA];[156delA]|\cite{Laks2019-dm})
parallel branches (\textit{p53-/-a} and \textit{p53-/-b}), each passaged over 60 generations (285 and 220 days, respectively) and sampled 7 times. A median of 1231 cells per passage were sequenced yielding a total of 6620, 7935, 9615 single cell genomes for each timeseries, respectively (\refsupfig{sfig:pdxExpDesCurves}A, \refsuptab{tab:omnibusMedians}). For each of \textit{p53 WT}, \textit{p53-/-a}, \textit{p53-/-b} we inferred single cell copy number profiles, constructed a phylogenetic tree to establish clonal lineages (see Methods, \refsupfig{sfig:heatmaps}) and measured clonal abundances as a function of time (\reffig{fig:cellline}A-C). 
Modeling the abundances with \textit{fitClone} (\refsuptab{stab:real_data_params}) revealed \textit{p53 WT} clonal trajectories consistent with small differences over the posterior distributions of fitness coefficients amongst four major clones (\reffig{fig:cellline}D). In contrast, p53 mutant branches each showed significant expansions of clones with aneuploid genotypes, where the founder diploid population was outcompeted. Relative to \textit{p53 WT}, rates of expansion of p53 mutant, aneuploid clones were significantly higher, leading to rapid depletion of diploid cells (\reffig{fig:cellline}D, p=6.72e-04). Pairwise difference $s_i-s_j$ ($\Delta$s) between clones in the \textit{fitClone} inference process was more extreme in the p53-/- lines relative to \textit{p53 WT} (\reffig{fig:cellline}E). This suggests that p53 mutation permits expansion of clones at higher rates, and that these clones have measurably higher fitness with positive selection coefficients.

The p53 mutant lines harbored 11 (size range 47 to 1,474 cells, median 204), and 10 (size range 158 to 997 cells, median 404) distinct clones for \textit{p53-/-a} and \textit{p53-/-b}, respectively. In each series the diploid founder clones, devoid of detectable copy number alterations, were systematically outcompeted by populations that had acquired at least one copy number alteration (\reffig{fig:cellline}F-L).  Some similarities in copy number events were observed during the two replicate timesseries. These included gains in chromosomes 13, 19p and 20, and losses on chromosomes 8p and 19q (\refsupfig{sfig:heatmaps}). However, by the end of the timeseries, the genotypes in the two lines had diverged considerably. Selective coefficients inferred by \textit{fitClone} were highest in clones with focal amplifications of known prototypic oncogenes in breast cancer\cite{Curtis2012-kq,Shah2012-vx,Nik-Zainal2016-tu,Cancer_Genome_Atlas_Network2012-ug} including in \textit{MDM4}, \textit{MYC} (\reffig{fig:cellline}F) and \textit{TSHZ2} and (\reffig{fig:cellline}J), in some cases on a whole genome doubled background. Clone A, the highest fitness clone in \textit{p53-/-a} (57\% of cells at last timepoint, 1+s=1.05 $\pm$ 0.09) exhibited a whole genome doubling event (18 chromosomes with four copies) and harboured a focal, high level amplification at the \textit{MDM4} locus on Chr1q (\reffig{fig:cellline}F). Clone G (27\% of cells at last timepoint, 1+s=1.03 $\pm$ 0.03), the next highest fitness clone in \textit{p53-/-a} remained diploid, with the exception of a focal high level amplification precisely at the \textit{MYC} locus on Chr8q (\reffig{fig:cellline}G). By contrast Clone K, here chosen as the reference clone for modeling, remained entirely diploid and exhibited a monotonically decreasing trajectory (from 90\% to 0\% of cells over the timeseries, \reffig{fig:cellline}H,I). In \textit{p53-/-b}, two clones exhibited non-neutral, positive selective coefficients (\reffig{fig:cellline}J). Clone D (52\% of cells at last timepoint,  1+s=1.05 $\pm$ 0.02) harbored a Chr20q single copy gain with a high level amplification at the \textit{TSHZ2} locus, while Clone E (35\% of cells at last timepoint, 1+s=1.05 $\pm$ 0.04) harbored a Chr4 loss, Chr19p gain/19q loss and Chr20q single copy gain (\reffig{fig:cellline}F). As seen in \textit{p53-/-a}, the ‘root’ Clone I that remained diploid was systematically outcompeted, diminishing from 68\% to 0\% abundance over the timeseries (\reffig{fig:cellline}K,L). 

We next tested whether the genomes of individual cells became progressively more aberrant over time and whether this correlated with measurements of clonal fitness. We estimated both sample and clone specific mutation rates at the copy number breakpoint and point mutation level (as previously described\cite{Laks2019-dm}). Both \textit{p53-/-a} and \textit{p53-/-b} exhibited increased mutations and breakpoints over time relative to \textit{p53 WT}  (\reffig{fig:cellrate}A-E). Cells accumulated 0.08 additional breakpoints (p$<$0.0001) and an average of 0.4 additional mutations (0.17 in \textit{p53-/-a} and 0.67 in \textit{p53-/-b}, p$<$0.001) per generation (\reffig{fig:cellrate}K,L, linear regression model, Supplemental methods), while the \textit{p53 WT} line accumulated 0.03 additional breakpoints per generation (\reffig{fig:cellrate}K). As this cell line was used as the reference for SNV detection in p53-/- cell lines, by definition no WT SNVs were analysed. Clone level distributions of breakpoints and mutations were positively correlated with inferred fitness coefficients in both p53-/- lines (\reffig{fig:cellrate}G,H,I,J, p=0.001) put not in \textit{p53 WT} (\reffig{fig:cellrate}D linear regression, p=0.5). Point mutation analysis in individual cells did not reveal any putative driver mutations associated with the fittest clones, selection pressure of mutations as measured by the dN/dS ratio\cite{Martincorena2017-jr} were not increased (\reffig{fig:cellrate}M), suggesting SNVs were not driving clonal expansions in this system.

These results indicate that the impact of genetic perturbation on selection can be measured and modeled in clonal populations. In particular, p53 mutation, known to be an early event in the evolution of many cancers\cite{ICGCTCGA_Pan-Cancer_Analysis_of_Whole_Genomes_Consortium2020-ef}, yields clonal expansions driven by whole genome, chromosomal and segmental aneuploidies, conferring quantitative fitness advantages over cells that maintain diploid genomes. Orthogonal analysis on mutation rates was consistent with expected increases as a function of inferred fitness coefficients, further corroborating biological relevance of \textit{fitClone} estimates. Modes of positive selection involving high level amplification of proto-oncogenes often seen in human breast cancer, and aneuploidies in general, suggest that in vitro genetic manipulations can induce fitness-enhancing genomic copy number changes consistent with etiologic roles in cancer. 


\subsection*{Clone-specific fitness estimates forecast clonal competition trajectories}
We next modeled clonal expansions observed during serial passaging of \textit{TP53} mutant PDX tumours to determine the predictive capacity of fitness coefficients (\reffig{fig:pdx}). We generated single cell genomes from 8 serial PDX transplants over 721 days from a Her2 positive (Her2+)  breast cancer with a \textit{TP53} p.A159P missense mutation, and contrasted this with 10 serial samples over 1002 days from a triple negative breast cancer (TNBC) PDX with a \textit{TP53} p.R213* non-sense mutation. A median of 907 single cell genomes were sequenced per passage for a total of 11,705 and 10,553 single cell genomes from the Her2+ and TNBC series, respectively (\refsuptab{tab:omnibusMedians}). Both series exhibited progressively higher tumour growth rates over time (\refsupfig{sfig:pdxExpDesCurves}D, \reffig{fig:pdx}F,J). Data were analysed as per the \emph{in vitro} lines described above and modeled with \texttt{fitClone} (\refsuptab{stab:real_data_params}). The Her2+ series exhibited 4 distinct clones ranging in size from 134 to 1,421 cells (median 319,  \reffig{fig:pdx}E), and the TNBC series exhibited 8 distinct clones with 18 to 680 cells (median 556, \reffig{fig:pdx}I). In the Her2+ model, clonal trajectories were consistent with selective coefficients with small relative differences in fitness (\reffig{fig:pdx}F,G, mean = 1.01$\pm$ sd = 0.01, Supplementary methods).
By contrast, the TNBC model trajectories resulted in a high positive selective coefficient for a minority of clones (\reffig{fig:pdx}J,K, mean = 1.03 $\pm$ sd = 0.11,  Supplementary Information). Consistent with increased dynamics in the TNBC series, we found an initial increase of 0.1 breakpoints per cell per generation in the first 4 passages (linear regression, p=0.03, $R^2$=0.49, \refsupfig{sfig:mutanalysis}A). After this initial increase the average number of breakpoints per cell remained constant (linear regression, p=0.8, $R^2$=0.08). In the Her2+ line we observed a small decrease of 0.04 copy number breakpoints per generation (linear regression, p=0.002, $R^2$=0.05). We note that Clone E in TNBC swept the population over the last 3 timepoints (\reffig{fig:pdx}J,K, n=541 over the timeseries) . \texttt{fitClone} selective coefficient 1+s was highest in Clone E (1+s= 1.08 $\pm$ 0.043), having grown from undetectable proportions in earlier timepoints to 58\% of cells by the end of the timeseries. Clone E also had the highest number of breakpoints with 12.8 additional copy number breakpoints per cell, relative to the reference clone C with the lowest (linear regression with coverage breadth, ploidy and cell cycle state as covariates, p $<$0.0001, $R^2$=0.364, \refsupfig{sfig:mutanalysis}C). 

We next asked whether the high predicted fitness of Clone E was a true indicator of positive selection through a physical clonal mixing and re-transplant experiment. Enforced clonal competition of higher fitness clones with lower fitness counterparts should result in re-emergence or fixation of high fitness clones, even when re-starting from a low population prevalence. To test this, we forward-simulated trajectories from \textit{fitClone} using the estimated selective coefficients (A=1.03 $\pm$ 0.16, B=1.00 $\pm$ 0.01, C=0.00,  D=1.00 $\pm$ 0.01, E=1.10 $\pm$ 0.10, F=1.01 $\pm$ 0.13, G=1.01 $\pm$ 0.02, H=1.02 $\pm$ 0.03) and starting clonal proportions of (A=0.00, B=0.07, C=0.25, D=0.51, E=0.02, F=0.00, G=0.08 H=0.07), derived by physically mixing cells from a late (X8) and an early (X3) passage  of the TNBC series (\reffig{fig:mix}C). We note that \textit{fitClone} assumes non-zero initial values for all clones, permitting growth of even exceedingly rare clones (Supplementary Information). We generated 47,000 trajectories from the model (\reffig{fig:mix}B) and found Clone E with the highest probability of fixation (0.32), followed by Clones A (0.17) and F (0.12). Fixation probabilities for the remainder of clones were low ($<$0.01 for H and D) or zero (for B, C, and G). We experimentally tested these predictions by initiating a new PDX with the remixed population, serially passaged over 4 timepoints (\reffig{fig:mix}C, \refsupfig{sfig:pdxExpDesCurves}A), and sequenced with DLP+ (7,839 single cell genomes, median 1,354.5 per library). Seven clones from the original timeseries were recapitulated in the mixture timeseries (all but Clone A) with between 26 to 499 (median 162) cells (\reffig{fig:mix}A,D). Clones with higher selective coefficients swept through the mixture timeseries by passage 4 (\reffig{fig:mix}D,E). Comparison of model fits of the original and mixture timeseries yielded similar posterior distributions for the majority of clones (\reffig{fig:mix}F,G). While Clone A was not detected in the mixture, Clone F emerged as a high fitness clone (1+s=1.10 $\pm$ 0.07). Notably, F was phylogenetically proximal to Clone E (\reffig{fig:mix}A) and thus likely represented a biologically similar population. In the last timepoint, the clade composed of clones E and F comprised 94\% of cells, outcompeting low-fitness clones and thus recapitulating \textit{fitClone} coefficients predictions and forecasts of clonal competition.

\subsection*{Clone-specific genotypes underpin clone specific gene expression programs}
We next profiled the impact of clone specific gene expression changes as a higher order representation of phenotypic properties. We tested if the genotypes of high fitness clones exhibited changes in their transcriptional program, with sc-RNAseq performed on matched aliquots of samples sequenced using DLP+ (Table S\ref{stab:tenx}). We applied a statistical model, clonealign\cite{Campbell2019-lp}, to map sc-RNAseq transcriptional profiles to their clone of origin, and to investigate gene dosage effects of copy number alterations on transcription. sc-RNAseq embeddings showed a dynamic pattern of global expression over time which tracked with clone assignments (\reffig{fig:exp}A), indicating co-variation of transcriptional properties with clonal abundance. Pairwise comparisons of clone-specific differential gene expression over the genome revealed spatially correlated distributions reflective of gene dosage effects (\refsupfig{sfig:diffexp}) at the level of whole chromosomes, whole arm and segmental aneuploidies.
Transcriptional profiles from late timepoints (when high fitness clones had grown in abundance) indicated copy number driven gene expression in each of the \textit{p53-/-a}, \textit{p53-/-b} and TNBC series (\reffig{fig:exp}B). We note that DLP+ and clonealign abundance measures were positively correlated across all libraries (Pearson correlation coefficient = 0.94, p$<$0.001, \reffig{fig:exp}C), consistent with clone specific gene dosage effects in the majority of libraries.  Examples from  in vitro and PDX series showed between 17 and 42\% of all differentially expressed genes have clone specific copy number differences (\reffig{fig:exp}D, FDR$<$0.01). In high fitness clones with high level amplifications as distinguishing features, we noted accompanying in cis clone-specific differential gene expression in \textit{PVT1} and \textit{MYC} in passage X57 of \textit{p53-/-a} (clone G with 9 copies of Chr8q), \textit{PFDN4} in passage X50 of \textit{p53-/-b} (clone D with 7 copies of Chr20q) and \textit{VCX3B} in passage X10 of TNBC (clone E with 8 copies of ChrXp) (\reffig{fig:exp}E, Table S\ref{stab:deviolin}). Together these data indicate that clonal genotypes driving high fitness trajectories are accompanied by changes in gene expression at both chromosomal and focal level copy number alterations.


\subsection*{Clonal competition and fitness costs of platinum resistance}
We tested how pharmacologic perturbation with cisplatin impacted the stability of the fitness landscape of the TNBC series. We generated a separate branch of the TNBC model where we administered cisplatin (2mg/kg, \textit{Q3Dx8} i.p. max) serially over four successive passages to induce drug resistance (\refsupfig{sfig:pdxExpDesCurves}C, Supplementary methods \ref{ssec:rx}). For each serially treated tumour, a parallel set of transplanted mice were left untreated, establishing corresponding drug ‘holiday’ samples (\reffig{fig:schematic}A). We coded the treated passages with ‘T’ and untreated with ‘U’, initialised by the X3 untreated (U) passage. The first treatment passage (\textit{X4 UT}) exhibited rapid tumour shrinkage ($>$50\% of initial size). However \textit{X5 UTT}, \textit{X6 UTTT} and \textit{X7 UTTTT} had progressively less response, indicating drug resistance and positive growth kinetics (\refsupfig{sfig:pdxExpDesCurves}E). Decomposing the growth dynamics over (\textit{X3 U}; \textit{X4 UT}; \textit{X5 UTT}; \textit{X6 UTTT}; \textit{X7 UTTTT}) into clonal trajectories with DLP+ analysis suggested sustained cisplatin treatment had inverted the fitness landscape. A new Clone R,  derived from the Clone A in the phylogeny, but with a distinct clonal genotype (fewer copies of \textit{MYC} and deletions at the \textit{RB1}, \textit{PRDM9} and \textit{NUDT15} loci (\reffig{fig:pdxrx}A,B)), swept to fixation comprising 48\% (\textit{X4 UT}), 98\% (\textit{X5 UTT}), 100\% (\textit{X6 UTTT}) and 100\% (\textit{X7 UTTTT}) of cells across the treated series (\reffig{fig:pdxrx}C). Notably, the high fitness clones E, H, G, D from the untreated series exhibited low fitness coefficients in the treatment series and were no longer detected (\reffig{fig:pdxrx}D, \refsupfig{sfig:rx}). Conversely, Clones A, B, and C, comprising a low fitness phylogenetic superclade, distinct from high fitness clones E and F in the untreated series, were the precursors to the resistant clone R  (\reffig{fig:pdxrx}E). Thus, cisplatin perturbation resulted in a near complete inversion of the fitness landscape.

We next asked whether the clonal dynamics in the presence of cisplatin were reversible by examining the drug holiday samples (\reffig{fig:schematic}A, \textit{X5 UTU}; \textit{X6 UTTU}; \textit{X7 UTTTU}). In the first drug holiday \textit{X5 UTU}, clonal composition reverted to consist predominantly of precursor clone B with 90\% abundance, and only 10\% abundance from clone R.  However, in \textit{X6 UTTU} and \textit{X7 UTTTU} no reversion was detected, and these populations consisted of  $>$99\% Clone R, similar to their on-treatment analogues. Thus, clonal competition in the absence of drug leads to clones derived from the A-B clade outcompeting clone R, and clone-specific cisplatin resistance thus has a fitness cost. Furthermore, the genotype specificity of reversion between \textit{X4 UT} to \textit{X5 UTU} indicates that the clonal dynamics can be attributed to selection of genomically defined clones with differential fitness. 

While transcription phenotype clusters tracked with genotypic clones in the untreated series (\reffig{fig:exp}), analysis of sc-RNAseq from the parallel treated samples indicated that phenotypes were also impacted by cisplatin (Pearson correlation of DLP+ clones and clonealign sc-RNAseq clones = 0.99, \refsupfig{sfig:rx_10x}A). Global transcription effects generating increased phenotypic volume\cite{Azizi2018-eb} (\reffig{fig:rxexp}A) and transcriptional velocity\cite{La_Manno2018-az} (\reffig{fig:rxexp}B and \refsupfig{sfig:rx_10x}B) were observed as a function of time on treatment.  As nearly all tumour cells were clone R from \textit{X5 UTT} onwards in the treatment series, we attributed transcriptional changes to phenotypic plasticity (\reffig{fig:rxexp}C,D and \refsupfig{sfig:rx_10x}C,D). Modified pathways included previously confirmed cisplatin-resistance metabolic processes such as oxidative phosphorylation \cite{lee2017myc}, TNFA signaling via NFKB \cite{lagunas2008nuclear,ito2015down,ryan2019targeting}, E2F targets \cite{zheng2020upregulation} and hypoxia \cite{lee2012hypoxia,mcevoy2015identifying,deben2018hypoxia,li2019erk}. Individual genes showing coordinated activity as a function of time on treatment from these pathways are shown in \reffig{fig:rxexp}E, relative to the untreated and holiday regimes as comparators, indicating the specificity of these pathway dysregulations to drug intervention. A minor component of the phenotypic volume is reversible (\reffig{fig:rxexp}A) on drug holiday, indicating only modest or partial reversibility of gene expression of clone R (\reffig{fig:rxexp}E, X5-$>$X6, X6-X7, Rx and Rx-H) for some genes (\textit{CEBP}, \textit{NDUF7}, \textit{MYC}).

Together, these data show the impact of cisplatin selective pressure on the starting tumour cell population is reversible while genomic clonal competition with precursor clones is still possible, but becomes fixed once the evolutionary bottleneck narrows and purifies the population. Once fixed, a minor component of the expression landscape remains reversible. Thus, cisplatin resistance occurred in phases - first dominated by selection on mixed populations, followed by transcriptional changes on a fixed genotype.

\ifcsdef{mainfile}{}{\bibliography{ref}}

\ifcsdef{mainfile}{}{\subfile{mainfigs}}

\ifcsdef{mainfile}{}{\subfile{tabs}}

\ifcsdef{mainfile}{}{\subfile{suppfigs}}

\ifcsdef{mainfile}{}{\subfile{supptabs}}


