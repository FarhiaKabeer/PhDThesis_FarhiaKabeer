%% The following is a directive for TeXShop to indicate the main file
%%!TEX root = diss.tex

\chapter{List of abbreviations}



% use \acrodef to define an acronym, but no listing
\acrodef{UI}{user interface}
\acrodef{UBC}{University of British Columbia}

% The acronym environment will typeset only those acronyms that were
% *actually used* in the course of the document
\begin{acronym}[PDX]
\acro{NER}{nucleotide excision repair}
\acro{PDX}{patient-derived xenograft}
\acro{NSG}{NOD/SCID/IL2r$\gamma^{\small{-/-}}$}
\acro{NRG}{ NOD/Rag1$^{\small{-/-}}$Il2r$\gamma ^{\small{-/-}}$}
\acro{QC}{Quality Control}
\acro{TNBC}{triple negative breast cancer}
\acro{ARC}{animal resource centre}
\acro{MTD}{Maximum tolerated dose}
\acro{CNA}{copy number aberration}
\acro{EGFR}{epidermal growth factor receptor (HER1)}
\acro{ER}{estrogen receptor-alpha}
\acro{ERBB2}{v-erb-b2 avian erythroblastic leukemia viral oncogene homolog 2 (HER2)}
\acro{FDR}{false discovery rate}
\acro{pCR}{pathologic complete response}
\acro{IHC}{immunohistochemistry}
\acro{FFPE}{formalin-fixed paraffin embedded}
\acro{FISH}{fluorescence in situ hybridization}
\acro{HER2}{human epidermal growth factor receptor 2 (ERBB2)}
\acro{INSR}{insulin receptor}
\acro{LBC}{lobular breast cancer}
\acro{LOH}{loss of heterozygosity}
\acro{NGS}{Next generation DNA sequencing}
\acro{PARP1}{poly(ADP-ribose) polymerase 1}
\acro{PBS}{phosphate-buffered saline}
\acro{PCR}{polymerase chain reaction}
\acro{PR}{progesterone receptor}


\end{acronym}

% You can also use \newacro{}{} to only define acronyms
% but without explictly creating a glossary
% 
% \newacro{ANOVA}[ANOVA]{Analysis of Variance\acroextra{, a set of
%   statistical techniques to identify sources of variability between groups.}}
% \newacro{API}[API]{application programming interface}
% \newacro{GOMS}[GOMS]{Goals, Operators, Methods, and Selection\acroextra{,
%   a framework for usability analysis.}}
% \newacro{TLX}[TLX]{Task Load Index\acroextra{, an instrument for gauging
%   the subjective mental workload experienced by a human in performing
%   a task.}}
% \newacro{UI}[UI]{user interface}
% \newacro{UML}[UML]{Unified Modelling Language}
% \newacro{W3C}[W3C]{World Wide Web Consortium}
% \newacro{XML}[XML]{Extensible Markup Language}
