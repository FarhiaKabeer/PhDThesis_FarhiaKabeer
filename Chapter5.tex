{\chapter{Characterization of cancer transcriptomics under drug perturbations in TNBC-PDX}

}
\label{ch:Chapter5}
\section{Synopsis}
From the previous chapter we know that the population changes with the tumor growth from one passage to another with and without the drugs. We see the shifts in the clones and some clones get selected and others disappear.
Now in this chapter, first of all we sought to determine whether under neutral conditions when there is no drug pressure what proportion of the transcriptomes might be copy number driven just due to the natural evolution of the copy number clones.This allows us to understand the stability of the gene expression in our timeseries models in the absence of drugs.
Next, we explored that on drug being introduced into the system, how much of the drug induced change in expression is dependent on the copy number and which. are independent.







\section{Results}
\subsection{Correlation coefficient between copy number and expression}

Each point is the proportion of DLP cells in a clone (x axis) versus the proportion of the tenx cells in the same clone. Then calculate the correlation coefficient for all clones. Once we have the clonealign results, we can very easily do this.

\subsection{Clonealign-Copy number dependent in-cis transcription patterns}
Clonealign approach was used to reveal clone-specific phenotypic properties. The serially passaged triple-negative breast cancer patient-derived xenografts from Chapter 4 as substrates.

\subsection{Cellular prevalance in sensitive and resistant clones based on single cell RNA sequencing}

\subsection{Copy number defined resistant clones showed distinct upregulated pathways as compared to sensitive clones}

\subsubsection{SA609-TNBC-FBI PDX up regulated pathways in resistant clone}


\subsubsection{SA535-TNBC-BRCA deficient PDX up regulated pathways in resistant clone}
Next we compared the emerging clone under drug pressure with the clone that could not survive the repeated drug exposures. We identified the following significantly up regulated pathways:
- Apoptosis, Epithelial mesenchymal transition, Hypoxia, mitotic spindle, MTORC1 signaling, MYC targets-V1, P53 pathway, TGF Beta signaling, TNFA signalling via NFKB, UV response up, UV response down, KRAS signaling, Angiogenesis, PI3K AKT MTOR signaling, unfolded protein response.


\subsubsection{SA1035-TNBC-APOBEC deficient PDX up regulated pathways in resistant clone}




\subsection{key Pathways and genes shared among the three TNBC PDX}




\subsection{Distinct genes monotonically increasing and decreasing with drug} 


