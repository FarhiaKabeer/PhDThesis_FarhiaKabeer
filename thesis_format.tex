\documentclass[12pt, a4paper]{article}
\setlength{\oddsidemargin}{0.5cm}
\setlength{\evensidemargin}{0.5cm}
\setlength{\topmargin}{-1.6cm}
\setlength{\leftmargin}{0.5cm}
\setlength{\rightmargin}{0.5cm}
\setlength{\textheight}{24.00cm} 
\setlength{\textwidth}{15.00cm}
\parindent 0pt
\parskip 5pt
\pagestyle{plain}

\title{Research Proposal}
\author{}
\date{}

\newcommand{\namelistlabel}[1]{\mbox{#1}\hfil}
\newenvironment{namelist}[1]{%1
\begin{list}{}
    {
        \let\makelabel\namelistlabel
        \settowidth{\labelwidth}{#1}
        \setlength{\leftmargin}{1.1\labelwidth}
    }
  }{%1
\end{list}}

\begin{document}
\maketitle

\begin{namelist}{xxxxxxxxxxxx}
\item[{\bf Title:}]
	Quantitative estimates of fitness in cancer using single cell population dynamics
\item[{\bf Author:}]
	Sohrab Salehi
\item[{\bf Supervisors:}]
	Professor Sohrab P. Shah \\
	Professor Alexandre Bouchard-C\^ot\'e
\item[{\bf Degree:}]
	MSc Bioinformatics, 2016, UBC
\end{namelist}

\section*{Agenda}

% How do we think cancer works
% What is mutation / heterogeneity 
% How does it relate to tumour evolution
% How has it been measured
% -- through bulk data
% -- indirectly, usnig clusters of mutations
% -- through the synonymous mutations
% -- Ideas transplanted from DE
% -- cell lines
% -- Xenografts
% -- CRISPER pool experiments

% What is its clinical relevance?
% Question?
 

\begin{itemize}
    \item Description of the samples in SA609
    \item Emphasise that we dont know if this is due to the CN changes or a SNV or some other change (including change in the environment) - remedy, repeat | killer experiment
    \item Add bulk?
    
\end{itemize}




\section*{Introduction} 

Main questions in the field
The usecases

Literature review:
Ways to measure selection, detect footprints of evolution
Classical methods

ka/ks rate
Methods based on the structure of phylogenetic trees 
Methods based on rate of nonsynonymous/synonymous mutations
limitations

Methods based on 1/f frequency [cite bigbang, and the new grapham paper]
Wright-Fisher ones [influenza paper]

Methods based on the different regimes of selection

Utility is prediction

In this section you should give some background to your
research area. What is the problem you are tackling, and why is it
worthwhile solving? Who has already done some work in this area,
and what have they achieved?

\section*{Proposed work} Now state explicitly the hypothesis you aim to
test. Make references to the items listed in the Reference section
that back up your arguments for why this is a reasonable
hypothesis to test, for example the work of d.
Explain what you expect will be accomplished by undertaking this
particular project.  Moreover, is it likely to have any other
applications?
 
 \subsubsection{Datasets}
 
 
 
\section*{Conclusions} 
Summary
Real-time monitoring of evolution of cancerous cells at the single cells copy number resolution is now feasible [1]. We have been working on adapting a diffusion approximation of the Wright Fisher (WF) model to infer the magnitudes of relative fitness in presence of genetic and/or drug-induced perturbations. Our current implementations support inference of up to a dozen clones (sub-populations). We have also worked on extensions to higher dimensions.

I propose the PhD dissertation to be organised as follows:
A chapter on Bayesian methods of analysing Wright Fisher (WF) diffusions (fitClone)
A chapter on application of fitClone to direct evolution experiments, i.e. feasible subset of xenografts lines SA532 and SA609, and isognetic cells lines SA1101
A chapter on adapting Bayesian methods of analysing WF onto double-barcoded CRISPR cell-lines

WF diffusion approximation - fitClone
Inference of relative fitness was modelled as a discretely observed state space model the dynamics of which is governed by the K-dimensional WF diffusion. An MCMC inference algorithm has been implemented from scratch in Python and subsequently optimised in C to support learning of selection coefficients and predicting future clonal abundance trajectories. It was profiled using synthetics datasets. A hierarchical model that supports multiple observation of the same process shows improved accuracy. A Gibbs-sampling based method has been partially developed as an extension to support higher dimensions. 
Exponential growth model
A model where clonal abundance at each time is a deterministic function of selection coefficients and the starting frequencies. This is a highly scalable model and works comparatively well to the WF-diffusion model in terms of parameter inference and as such is a practical candidate for higher dimensional problems, e.g., CRISPR-pool depletion.
Biological substrates
New datasets are being procured as part of the direct evolution experiments. These include the deep-spanning tree xenograft lines and isogenic hTERT cell-lines. Currently the two xenograft lines have 6 consecutive timepoints each and are planned to be extended to 10. Parallel branches are to come later. 
Three parallel hTERT cells-lines (wild-type and p53-null) have been passaged for 60 generations. These are sparsely sampled (the longest having three timepoints), however up to 6 have been planned.
Double-barcoded CRISPR libraries
Double-barcoded CRISPR libraries allow tracking a cell and all its direct progeny. About 60 libraries, a mix of xenografts and cell-lines have been generated to investigate the effects of modified genes on cell-survival in different genomic backgrounds. We posit that this could be modelled using a branching process and propose to adapt the fitClone machinery or the framework developed for the exponential growth model to model the trajectories of cells and ascertain selection coefficients [2].  
			
					
References						
					
[1] Zahn H, Steif A, Laks E, Eirew P, VanInsberghe M, Shah SP, Aparicio S, Hansen CL. Scalable whole-genome single-cell library preparation without preamplification. Nature methods. 2017 Feb;14(2):167.

Bansaye V, Méléard S. Some stochastic models for structured populations: scaling limits and long time behavior. arXiv preprint arXiv:1506.04165. 2015 Jun 12.

 
\section*{Method}
In this section you should outline how you intend to go
about accomplishing the aims you have set in the previous
section. Try to break your grand aims down into small,
achievable tasks. Try to estimate how long you will
spend on each task, and draw up a timetable for each
sub-task.\cite{jorde1999estimating}



% For some reason overeleaf can't find the reference file. Tex it can though...
\bibliographystyle{vancouver}
\bibliography{../fitclone/ref} 

\end{document}

