%% The following is a directive for TeXShop to indicate the main file
%%!TEX root = diss.tex

%% https://www.grad.ubc.ca/current-students/dissertation-thesis-preparation/preliminary-pages
%% 
%% LAY SUMMARY Effective May 2017, all theses and dissertations must
%% include a lay summary.  The lay or public summary explains the key
%% goals and contributions of the research/scholarly work in terms that
%% can be understood by the general public. It must not exceed 150
%% words in length.

 \chapter{Lay Summary}

\textbf{The lay or public summary explains the key goals and contributions of
the research\slash{}scholarly work in terms that can be understood by the
general public. It must not exceed 150 words in length.}

% This text need to change
\textbf[{needs correction}] Cancer cells become resistant to drugs via a natural selection process similar to antibiotic resistance. Learning
how to rationally combine treatments to avoid drug resistance is a challenge. Triple-negative breast cancer
(TNBC) is an example of a difficult to treat disease - there are no `targeted' therapies and we do not fully
understand what makes TNBC tumours different between patients. Dr. Aparicio and his colleagues have
developed new technologies and approaches to probe single cancer cells with unprecedented fidelity, and aim to
find new therapies by exploiting the observation that TNBC may differ in the way that the tumour genomes become unstable. This work in TNBC will serve as a model to solve similar problems in other cancers.
previous research
The notion that cancers are mosaics, composed of various cellular clones (groups of cells related to each other
via cell division) of malignant cells with different characteristics, has received attention for several decades. Cell
variation happens at several scales:spatially, within individual cancers; within individual patients over time and
space;and between individual patients. Despite the fundamental importance of cancer clonal dynamics, much of
our understanding of mechanisms and treatments is based on single snapshots in time or limited resolution
sampling. Until recently, clonality studies in cancer have been hampered by the lack of methods to identify clonal
structure. We have made progress on this front by developing an accessible and scalable method for genomic
analysis of single tumour cells, combined with new computational methods to infer and visualize clonal
populations. These will allow for better diagnosis and treatment of cancer in the context of our proposed study.

project description
Our project will learn how to forecast which combinations of treatments will be effective for patients with triple
negative breast cancer (TNBC) and how to find new drug targeting approaches. The treatment problem we are
trying to solve arises because TNBC tumours are not uniform, but are composed of mosaics of cells that have
different capacities to evolve and respond to therapy, a property called cellular fitness. Our project will focus on
discovering determinants of fitness in TNBC, a group of cancers that do not have well defined `targeted'
therapeutic options. We propose new methods to sequence the DNA, RNA and epigenome of thousands of single
cells from patient TNBC tumours grown in mouse `avatars', in order to learn which properties are associated with
fitness to drugs, taken in combination. Ultimately, we will analyze TNBC patient biopsies to directly associate their
genotypes/epigenotypes with subsequent behavior and biopsies from trials of new drug agents for TNBC.

 %impact and relevance statement
%Our overall goal is to reduce the burden of untreatable disease by learning how to use combinations of drugs
%more effectively to reduce drug resistance and relapse. 
%TNBC tumours are not uniform between patients, and
%therefore we have no ‘targeted’ therapies; however, we may find these by exploiting the observation that TNBCs
%differ in the way that the tumour genomes become unstable.
%We will determine the cellular fitness associated with
%specific mutations or epigenomic signatures at the single-cell level from TNBC tumours, although the methods
%and approaches will be generally applicable to other cancers. We shall employ the principles of natural selection in patients during the course of their disease to inform on novel drug targets, evaluate targets of selection, and
%guide diagnostics development.The ability to resolve genomic/epigenomic variation and phenotypes at single cell
%resolution is at the leading edge of the field and will fundamentally change how new drugs are discovered

