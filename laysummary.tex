%% The following is a directive for TeXShop to indicate the main file
%%!TEX root = diss.tex

%% https://www.grad.ubc.ca/current-students/dissertation-thesis-preparation/preliminary-pages
%% 
%% LAY SUMMARY Effective May 2017, all theses and dissertations must
%% include a lay summary.  The lay or public summary explains the key
%% goals and contributions of the research/scholarly work in terms that
%% can be understood by the general public. It must not exceed 150
%% words in length.

 \chapter{Lay Summary}

Cancer cells become resistant to drugs via a natural selection process similar to antibiotic resistance. Cancers are composed of various cellular clones (groups of cells related to each other via cell division) of malignant cells with different characteristics. Cell variations happens at several scales; spatially, within individual cancers; within individual patients over time and space; and between individual patients. In this dissertation, first, we present the optimizations and effects of tumor digestion into single cells. Then, triple negative breast cancer patient's tumours were grown in mouse `avatars' and serially passaged over generations with and without chemotherapy treatment. We found that by exposing the tumors repeatedly to the same chemotherapy, it favours some predictable clonal population to expand and some cells will disappear, ultimately leading to drug resistance. Finally, we explored these resistant versus sensitive cells by using new methods to sequence the DNA and RNA of thousands of single cells.



 %impact and relevance statement
%Our overall goal is to reduce the burden of untreatable disease by learning how to use combinations of drugs
%more effectively to reduce drug resistance and relapse. 
%TNBC tumours are not uniform between patients, and
%therefore we have no ‘targeted’ therapies; however, we may find these by exploiting the observation that TNBCs
%differ in the way that the tumour genomes become unstable.
%We will determine the cellular fitness associated with
%specific mutations or epigenomic signatures at the single-cell level from TNBC tumours, although the methods
%and approaches will be generally applicable to other cancers. We shall employ the principles of natural selection in patients during the course of their disease to inform on novel drug targets, evaluate targets of selection, and
%guide diagnostics development.The ability to resolve genomic/epigenomic variation and phenotypes at single cell
%resolution is at the leading edge of the field and will fundamentally change how new drugs are discovered

