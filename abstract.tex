%% The following is a directive for TeXShop to indicate the main file
%%!TEX root = diss.tex


\chapter{Abstract}
Cancer is an ecosystem of genetically diverse evolving clones, which emerge in time and space as a result of genomic and non-genomic instability. Consequently, clonal evolution can derive selection in cancer, impacting etiology and clinically important behaviour such as drug resistance. Progress in defining clonal fitness landscapes has been impeded by a lack of controlled perturbation experiments with timeseries observations over realistic intervals at single cell resolution. Our goal is to reveal the intricacies of cancers to explore the effects of spatial and temporal heterogeneity on the population dynamics. This dissertation is focused on the development of triple negative breast cancer (TNBC) patient derived xenografts (PDXs) timeseries with and without drug perturbations. Therefore, implementation of computational methods, to decode breast cancer heterogeneity, clonal evolution and fitness. First, we established PDXs with a distinct set of mutational background and performed in vivo chemotherapy screens to assess the population responses. Subsequently, we optimized methods of tumor dissociations, to determine the effects of enzymatic digestion time and temperature on gene expression. 
Next, we developed an experimental and mathematical framework for predicting the likely trajectories of clones. We studied clonal expansions observed during serial passaging of TP53 mutant human breast cancer cells. We exhibit the nature of clonal dynamics induced by pharmacologic perturbation with a quantitative fitness model and serial single cell whole genome and transcriptome measurements over months to years. The repeated observations of sub-populations defined by copy number alterations (CNAs) show that markers closely linked to the CNAs or the CNAs themselves behave deterministically. Finally, we presented the close correlation between copy number variation and differential gene expression. It is critical to elucidate the relationship between copy number variation and gene expression for prevention, diagnosis and treatment of cancer.

Together, our findings advance the interpretation of how multi-faceted drug resistance mechanisms shape the etiology and cellular fitness of human cancers. Thus, cisplatin resistance occurred in phases, first, dominated by clonal selection on mixed populations, followed by transcriptional plasticity on a fixed genotype. Comprehensive measurements and analysis of clonal structure in heterogeneous tumors allow improved classification and combinatorial treatments of subpopulations.








% Consider placing version information if you circulate multiple drafts
%\vfill
%\begin{center}
%\begin{sf}
%\fbox{Revision: \today}
%\end{sf}
%\end{center}
