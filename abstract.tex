%% The following is a directive for TeXShop to indicate the main file
%%!TEX root = diss.tex

%% 350 words
\chapter{Abstract}
Cancer is an ecosystem of genetically diverse evolving clones, which emerge in time and space as a result of genomic and non-genomic instability. Consequently, clonal evolution can derive selection in cancer, impacting etiology and clinically important behaviour such as drug resistance. However, progress in defining clonal fitness landscapes, especially those defined by copy number alterations (CNA), has been impeded by a lack of controlled perturbation experiments with timeseries single cell sampling of polyclonal populations at single cell resolution. This dissertation is focused on the development of serial passaging of TP53 mutant human breast cancers, including three triple negative breast cancer (TNBC) patient derived xenografts (PDXs) timeseries, with and without drug perturbations. Our goal is to reveal the intricacies of cancers to explore the effects of spatial and temporal heterogeneity on the population dynamics.
First, we performed \textit{in vivo} chemotherapy screens to assess the baseline responses. Subsequently, we optimized methods of tumor dissociations, and determined the effects of digestion time and temperature on single cell gene expression. Next, by using a new probabilistic fitness model developed to ascribe quantitative selective coefficients to individual CNA defined clones, we found that in three different serially passaged cisplatin-treated patient derived xenograft (PDX) transplant lines of primary TNBC, the untreated fitness landscape was inverted. Notably, drug resistant clones reproducibly emerge from phylogenetic lineages of low fitness clones, and high fitness clones were eradicated. Moreover, drug treatment `holiday' time points indicate reversibility, suggesting both a fitness cost of the drug resistant genotypes and association of fitness inversion with drug selection. Finally, we presented the close correlation between copy number variation and differential gene expression, by exploring single cell RNA sequencing data on the same independent serial samples. We found that majority of genomic variations generated a direct effect on gene transcriptional level and were clustered in chemotherapy resistance related pathways, which suggests the involvement of CNA in pathophysiology of drug resistance.

These findings define approaches to comprehensive measurements and analysis of clonal structure in heterogeneous tumors and linking clonal fitness to CNA, and its related transcriptional changes, therefore, advance interpretation of polyclonal resistance to therapy.





% Consider placing version information if you circulate multiple drafts
%\vfill
%\begin{center}
%\begin{sf}
%\fbox{Revision: \today}
%\end{sf}
%\end{center}
