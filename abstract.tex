%% The following is a directive for TeXShop to indicate the main file
%%!TEX root = diss.tex


\chapter{Abstract}
Tumour fitness landscapes underpin selection in cancer, impacting etiology, evolution and response to treatment. Progress in defining fitness landscapes has been impeded by a lack of controlled perturbation experiments with timeseries observations over realistic intervals at single cell resolution.
%and a reliance on bulk sequencing technology with limited resolution for identifying clonal populations.
We studied the nature of clonal dynamics induced by  pharmacologic perturbation with a quantitative fitness model and serial single cell genome and transcriptome measurements over months to years. 
Our model enables prediction of clone-specific growth potential, forecasting competitive clonal dynamics over time and ascribing  quantitative selective coefficients to individual cancer clones. 

In addition, predicted selective coefficients from patient derived xenografts accurately forecasted clonal competition dynamics, validated with timeseries sampling of experimentally engineered mixtures of low and high fitness clones.
In cisplatin-treated patient derived xenografts, the fitness landscape was inverted in a time-dependent manner, whereby a drug resistant clone emerged from low fitness clones were selected under treatment and high fitness clones were eradicated. 
Clonal competition mediated early drug reversibility, whereas late dynamics were platinum associated with transcriptional plasticity  of  a  single  resistant  clone variation was evident after clone fixation. 
Together, our findings advance the interpretation of how mutations and multi-faceted drug resistance mechanisms shape the etiology and cellular fitness of human cancers.
%suggested:
%In cisplatin-treated patient derived xenografts, we observed a time-dependent multi-faceted drug 
%resistance process, first driven by selection and eradication of high fitness clones in the untreated %setting and then driven by clonal outgrowth of a resistance untreated high fitness clones and emergence 
%and eventual dominance of cisplatin resistant clones from precursor low fitness clones.
%Paradoxically, cisplatin resistant clones were less fit than their precursors when treatment was withdrawn before fixation. Thus clonal competition underpins early drug reversibility, whereas transcriptional variation underpins later resistance after clone fixation. Our findings show that both clonal fitness and drug resistance can be decoded from serial single cell measurements of genome and transcriptome.
%Our findings outline causal mechanisms with implication for interpreting how mutations and drug treatment
%shape the etiology and cellular fitness of human cancers.


%%% Old abstract
%Breast cancer is an ecosystem of genetically diverse evolving clones, which emerges in time and space. 
%Genomic instability results in diversity, which promotes clonal evolution, and clonal dynamics, which in turn underpins clinically important behaviour such as drug resistance and metastasis. 
%Our goal is to understand clonal evolution and develop experimental and mathematical framework for predicting the likely trajectories of clones in patients. 
%The stochastic nature of mutation accumulation and subsequent selection dictates that the tumour growth process is inherently random.
%By single cell whole genome sequencing and mathematical approaches to quantifying dynamics and fitness, we are mapping the relationships between clones and their fitness characteristics.
%Here we present fitClone, a statistical framework that infers the magnitudes of relative fitness in presence of genetic perturbations measured over timeseries, and predicts plausible future trajectories, supporting inference in up to hundreds of concurrent clones. We demonstrate our methods utility over serially transplanted PDX and cell lines. 
%Directed evolution experiments serial xenoengraftment systems and our new statistical model of quantification enable real-time observation of cancer evolution.
%Advances in single cell sequencing technology in turn allow for measurement of clonal dynamics of cancer with unprecedented accuracy.
%Quantitative reasoning about the future trajectory, dominance, and depletion of subpopulations in a tumour will be invaluable in predicting patient response to treatment, including rational drug therapy. 



% Consider placing version information if you circulate multiple drafts
%\vfill
%\begin{center}
%\begin{sf}
%\fbox{Revision: \today}
%\end{sf}
%\end{center}
